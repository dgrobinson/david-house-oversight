\documentclass{article}
\title{Additional Material Submitted for the Record: \\
Federal Regulations that Hold Back Government Web Sites}
\author{David G. Robinson\\
Associate Director, Center for Information Technology Policy\\
Princeton University}
\date{Date}
\begin{document}
   \maketitle
 As Chairman Towns noted at last week's hearing, a thicket of regulations 
holds back the full development and usefulness of many federal web sites. Per 
his request, I write to provide background on the issue. This brief note is 
designed to help the Committee consider possible oversight activity; I study 
these issues and would be happy to provide more detail as needed.


Test mod.

The site \texttt{webcontent.gov}, where federal web managers share their best 
practices, is a fascinating window into the obstacles they face. An online 
compliance checklist for designers of federal Web sites identifies more than 
twenty different regulatory regimes with which all public federal Web sites 
must comply.  Ranging from privacy and usability to FOIA compliance to the 
demands of the Paperwork Reduction Act and, separately, the Government 
Paperwork Elimination Act, each of these requirements may, considered on its 
own, be a thoughtfully justified federal mandate. But their cumulative effect 
is to place federal Web designers in a compliance minefield that makes it hard 
for them to avoid breaking the rules—while diverting energy from innovation 
into compliance.
\bibliographystyle{plain}
\bibliography{paper}
\end{document}
